\startcomponent c_alphabet
\mullertitle{alphabet}{Chapter I.}{The Alphabet.}

\s Sanskrit is properly written with the Devanāgarī alphabet; but the Bengali, %
% second ed
Grantha, %
%
Telugu, and other modern Indian alphabets are commonly employed for writing
Sanskrit in their respective provinces.

\startnotes

Note---\tl{Devaṅāgarī} means the \tl{Nāgarī} of the gods, or, possibly, of the
Brāhmans. A more current style of writing, used by Hindus in all common
transactions where Hindi is the language employed is called simply \tl{Nāgarī}.
Why the alphabet should have been called \tl{Nāgarī} is unknown. If derived from
\tl{nagara}, \eng{city}, it might mean the art of writing as first practised in
cities \panini{4.2.128}. No authority has yet been adduced from any ancient
author for the employment of the word \tl{Devanāgarī}. In the
\emph{Lalitavistara} (a life of Buddha, translated from Sanskrit into Chinese 76
{\sc a.d.}), where a list of alphabets is given, the \tl{Devanāgarī} is not
mentioned, unless it be intended by the \tl{Deva} alphabet. (See {\em History of
Ancient Sanskrit Literature}, p.\ 518.) Al-Biruni, in the 11th century, speaks
of the \tl{Nagara} alphabet as current in Malva (Reinaud, {\em Mémoire sur
l'Inde}, p.\ 298).

% second ed
\tl{Beghrām} (\tl{bhagārāma}, \eng{abode of the gods}) is the native name of one
or more of the most important cities founded by the Greeks, such as Alexandria
ad Caucasum or Nicæa. (See Mason's \emph{Memoirs} in Prinsep's
\emph{Antiquities}, ed.\ Thomas, vol.\ 1. pp.\ 344–350.) Could Devanāgarī have
been meant as an equivalent of Beghrāmi?
%

No inscriptions have been met with in India anterior to the rise of Buddhism.
The earliest authentic specimens of writing as the inscriptions of king
Priyadarśi or Aśoka, about 250 {\sc b.c}. These are written in two different
alphabets. The alphabet which is found in the inscription of Kapurdigiri, and
which in the main is the same as that of the Arianian coins, is written from
right to left. It is clearly of Semitic origin, and most closely connected with
the Aramaic branch of the old Semitic or Phoenician alphabet. The Aramaic
letters, however, which we know from Egyptian and Palmyrenian inscriptions, have
experienced further changes since they served as the model for the alphabet of
Kapurdigiri, and we must have recourse to the more primitive types of the
ancient Hebrew coins and of the Phoenician inscriptions in order to explain some
of the letters of the Kapurdigiri alphabet.

But while the transition of the Semitic types into this ancient Indian alphabet
can be proved with scientific precision, the second Indian alphabet, that which
is found in the inscription of Girnar, and which is the real source of all other
Indian alphabets, as well as of those of Tibet and Burma, has not as yet been
traced back in a satisfactory manner to any Semitic prototype (Prinsep's {\em
Indian Antiquities by Thomas}, vol.\ 2, p.\ 42). To admit, however, the
independent invention of a native Indian alphabet is impossible. Alphabets were
never invented, in the usual sense of that word. They were formed gradually, and
purely phonetic alphabets always point back to earlier, syllabic or ideographic,
stages. There are no such traces of the growth of an alphabet on Indian soil;
and it is to be hoped that new discoveries may still bring to light the
intermediate links by which the alphabet of Girnar, and through it the modern
Devanāgarī, may be connected with one of the leading Semitic alphabets.

\stopnotes

\s Sanskrit is written from left to right.

\startnotes

Note---\tl{Saṁskṛta} ({\ss संस्कृत}) means what is rendered \eng{fit} or
\eng{perfect}. But \eng{Sanskrit} is not called so because the Brāhmans, or
still less, because the first Europeans who became acquainted with it,
considered it the most perfect of all languages. \tl{Saṁskṛta} meant what is
rendered fit for sacred purposes; hence \eng{purified}, \eng{sacred}. A vessel
that is purified, a sacrificial victim that is properly dressed, a man who has
passed through all the initiatory rites or \tl{saṁskāras}; all these are called
\tl{saṁskṛta}. Hence the language which alone was fit for sacred acts, the
ancient idiom of the Vedas, was called \tl{Saṁskṛta}, or the sacred language.
The local spoken dialects received the general name of \tl{prākṛta}. This did
not mean originally \eng{vulgar}, but \eng{derived}, \eng{secondary},
\eng{second-rate}, literally \quote{what has a source or type,} this source or
type (\tl{prakṛti}) being the Saṁskṛta or sacred language. (See Vararuci's {\em
Prākṛta-Prakāśa}, ed.\ Cowell, p.\ xvii.)

% second ed
The former explanation of \tl{prākṛta} in the sense of \quote{the natural,
original continuations of the old language (\tl{bhāṣā}),} is untenable, because
it interpolates the idea of continuation. If \tl{prākṛta} had to be taken in
the sense of \quote{original and natural,} a language so called would mean, as
has been well shown by D'Alwis ({\em An Introduction to Kaccāyana's Grammar},
p.\ lxxxix), the original language, and \tl{saṁskṛta} would then have to be
taken in the sense of \quote{refined for literary purposes.} This view, however,
of the meaning of these two names, is opposed to the view of those who framed
the names, and is rendered impossible by the character of the Vedic language.
%

\stopnotes

\s In writing the Devanāgarī alphabet, the distinctive portion of each letter is
written first, then the perpendicular, and lastly the horizontal line. Ex.\ X,
X, \tr{क}{k}; X, X, \tr{ख}{kh}; X, X, \tr{ग}{g}; X, X, \tr{घ}{gh}; X, \tr{ङ}{ṅ},
\&c.

Beginners will find it useful to trace the letters on transparent paper till
they know them well and can write them fluently and correctly.

\s The following are the sounds which are represented in the Devanāgarī
alphabet:

% TODO: table is far too big!
\starttabulate[|p(1.5cm)l|p(0.7cm)c|p(1cm)c|p(0.7cm)c|p(1cm)c|p(0.6cm)c|p(0.8cm)c|p(1cm)c|p(0.6cm)c|p(0.8cm)c|p(1.7cm)c|]
  \HL
  \NC \NC Hard, (tenues.) \NC Hard and aspirated, (tenues aspiratæ.)
  \NC Soft, (mediæ.) \NC Soft and aspirated, (mediæ aspiratæ.)
  \NC Nasals. \NC Liquids. \NC Sibilants. \NC Vowels
  Long~~Short \NC \NC Diphthongs.\NC \NR
  \HL

  \NC 1.\,Gutturals, \NC \tr{क}{k} \NC \tr{ख}{kh} \NC \tr{ग}{g} \NC \tr{घ}{gh}
  \NC \tr{ङ}{ṅ} \NC \tr{ह}{h}\footnote[Footb]{\tr{ह}{h} is not properly a
  liquid, but a soft breathing.} \NC {\ss X}\footnote{The signs for the guttural
  and labial sibilants have become obsolete, and are replaced by the two dots
  \tr{ः}{ḥ}.} (\tl{χ}) \NC \tr{अ}{a} \NC \tr{आ}{ā} \NC \tr{ए}{e} \tr{ऐ}{ai} \NC
  \NR

  \NC 2.\,Palatals, \NC \tr{च}{c} \NC \tr{छ}{ch} \NC \tr{ज}{j} \NC \tr{झ}{jh}
  \NC \tr{ञ}{ñ} \NC \tr{य}{y} \NC \tr{श}{ś} \NC \tr{इ}{i} \NC \tr{ई}{ī} \NC \NC
  \NR

  \NC 3.\,Linguals, \NC \tr{ट}{ṭ} \NC \tr{ठ}{ṭh} \NC \tr{ड}{ḍ}\footnote[a]{In the
  Veda \tr{ड}{ḍ} and \tr{ढ}{ḍh}, if between two vowels, are in certain schools
  written \tr{ळ}{ḷ} and \tl{ळ्ह}.} \NC \tr{ढ}{ḍh} \NC \tr{ण}{ṇ} \NC \tr{र}{r} \NC
  \tr{ष}{ṣ} \NC \tr{ऋ}{ṛ} \NC \tr{ॠ}{ṝ} \NC \tr{ओ}{o} \tr{औ}{au} \NC \NR

  \NC 4.\,Dentals, \NC \tr{त}{t} \NC \tr{थ}{th} \NC \tr{द}{d} \NC \tr{ध}{dh}
  \NC \tr{न}{n} \NC \tr{ल}{l} \NC \tr{स}{s} \NC \tr{ऌ}{ḷ} \NC (\tr{ॡ}{ḹ}) \NC
  \NR

  \NC 5.\,Labials, \NC \tr{प}{p} \NC \tr{फ}{ph} \NC \tr{ब}{b} \NC \tr{भ}{bh}
  \NC \tr{म}{m} \NC \tr{व}{v}\footnote{\tr{व}{v} is sometimes called {\em
  dento||labial}.} \NC {\ss X} (\tl{φ}) \NC \tr{उ}{u} \NC \tr{ऊ}{ū}
  \NC \NC \NR
  \HL
\stoptabulate

\startnarrower
Unmodified nasal or \tl{anusvāra}, \tr{ः}{ṁ} or \tr{ँ}{X}.

Unmodified sibilant or \tl{visarga}, \tr{ः}{ḥ}.
\stopnarrower

Students should be cautioned against using the Roman letters instead of the
Devanāgarī when beginning to learn Sanskrit. The paradigms should be impressed
on the memory in their real and native form, otherwise their first impressions
will become unsettled and indistinct. After some progress has been made in
mastering the grammar and in reading Sanskrit, the Roman alphabet may be used
safely and with advantage.

\s There are fifty letters in the Devanāgarī alphabet: thirty-seven consonants
and thirteen vowels, representing every sound of the Sanskrit language.

\s One letter, the long \tr{ॡ}{ḹ}, is merely a grammatical invention; it never
occurs in the spoken language.

\s Two sounds, the guttural and labial sibilants, are now without distinctive
representatives in the Devanāgarī alphabet. They are called \tl{jihāmūlīya}, the
tongue-root sibilant, formed near the base of the tongue; and \tl{upadhmānīya},
i.e.\ {\em afflandus}, the labial sibilant. They are said to have been
represented by the signs XXX (called \tl{vajrākṛti}, having the shape of the
thunderbolt) and XXX (called \tl{gajakumbhākṛti}, having the shape of an
elephant's two frontal bones). (See Vopadeva's {\em Sanskrit Grammar}, 1.\ 18;
{\em History of Ancient Sanskrit Literature}, p.\ 508.) Sometimes the sign XXX,
called \tl{ardhavisarga}, \eng{half-visarga}, is used for both. But in common
writing these two signs are now replaced by the two dots, the \tl{dvivindu},
{\ss ः}, (\tl{dvi}, two, \tl{vindu}, dot) properly the sign of the unmodified
visarga.
% second ed
The old sign of the visarga is described in the {\em Kātantra} as like the
figure \tr{४}{4}; in the {\em Tantrābhidhāna} as like two \tr{ठ}{ṭh}'s. (See
Princep, {\em Indian Antiquities}, vol.\ 1.\ p.\ 75.)
%

\s There are five distinct letters for the five nasals, \tr{ङ्‌}{ṅ}, \tr{ञ्‌}{ñ},
\tr{ण्‌}{ṇ}, \tr{न्‌}{n}, \tr{म्‌}{m}, as there were originally five distinct signs
for the five sibilants. When, in the middle of words, these nasals are followed
by consonants of their own class, (\tl{ṅ} by \tl{k}, \tl{kh}, \tl{g}, \tl{gh};
\tl{ñ} by \tl{c}, \tl{ch}, \tl{j}, \tl{jh}; \tl{ṇ} by \tl{ṭ}, \tl{ṭh}, \tl{ḍ},
\tl{ḍh}; \tl{n} by \tl{t}, \tl{th}, \tl{d}, \tl{dh}; \tl{m} by \tl{p}, \tl{ph},
\tl{b}, \tl{bh},) they are often, for the sake of more expeditious writing,
replaced by the dot, which is properly the sign of the unmodified nasal or
anusvāra. Thus we find

\startnarrower
  {\ss अंकिता} instead of \tr{अङ्किता}{aṅkitā}

  {\ss अंचिता} instead of \tr{अञ्चिता}{añcitā}

  {\ss कुंदिता} instead of \tr{कुण्डिता}{kuṇḍitā}

  {\ss नंदिता} instead of \tr{नन्दिता}{nanditā}

  {\ss कंपिता} instead of \tr{कम्पिता}{kampitā}
\stopnarrower

The pronunciation remains unaffected by this style of writing. {\ss अंकिता} must
be pronounced as if it were written \tr{अङ्किता}{aṅkitā}, \&c.

The same applies to final \tr{म्‌}{m} at the end of a sentence. This too, though
frequently written and printed with the dot above the line, is to be pronounced
as \tr{म्‌}{m}. {\ss अहं}, \eng{I}, is to be pronounced like \tr{अहम्‌}{aham}. (See
preface to {\em Hitopadeśa}, in M.\ M.'s {\em Handbooks for the Study of
Sanskrit}, p.\ viii.)

\startnotes

Note---According to the Kaumāras final \tr{म्‌}{m} {\em in pausā} may be
pronounced as anusvāra; cf.\ {\em Sārasvatī-prakriyā}, ed.\ Bombay,
1829,\footnote{This edition, which has lately been reprinted, contains the
text---ascribed either to Vāṇī herself, i.e.\ Sarasvatī, the goddess of speech
(MS Bodl.\ 386), or to Anubhūti-svarūpa-āchārya, whoever that may be---and a
commentary. The commentary printed in the Bombay editions is called {\ss महीघरी},
or in MS Bodl.\ 382 {\ss मैदासी}, i.e.\ {\ss महीदासी}. In MS Bodl.\ 382
Mahīdhara or Mahīdāsabhaṭṭa is said to have written the {\em Sārasvata} in order
that his children might read it, and to please Īśa, \eng{the Lord}. The date
given is 1634, the place Benares (Śivarājadhanī).} pp.\ 12 and 13.
{\ss कौमारास्त्ववसानेऽप्पनुस्वारमिच्छंति । अवसाने वा । अवसाने सकारस्यानुस्वारो भवति २३. । देवं । देवम्‌ ।।}
The Kaumāras are the followers of Kumāra, the reputed author of the Kātantra or
Kalāpa grammar. (See Colebrooke, {\em Sanskrit Grammar}, preface; and page 315,
note.) Śarvavarman is sometimes quoted by mistake as the author of this grammar,
and an unnecessary distinction is made between the Kaumāras and the followers of
the Kalāpa grammar.

\stopnotes

\s Besides the five nasal letters, expressing the nasal sound as modified by
guttural, palatal, lingual, dental, and labial pronunciation, there are still
three nasalized letters, the {\ss य्ँ‌}, {\ss ल्ँ}, {\ss व्ँ}, or {\ss य्ं}, {\ss ल्ं},
{\ss व्ं}, XXX, XXX, XXX, which are used to represent a final \tr{म्‌}{m}, if
followed by an initial \tr{य्‌}{y}, \tr{ल्‌}{l}, \tr{व्‌}{v}, and modified by
the pronunciation of these three semivowels \panini{8.4.59}.

% TODO: not correct sanskrit transcription
\starttabulate[|k0l|i0l|]
  \NC Thus~ \NC instead of \tr{तं याति}{taṁ yāti} we may write \tr{तय्याँति}{taX yāti};\NC\NR
  \NC\NC instead of \tr{तं लभते}{taṁ labhate} we may write \tr{तल्लँभते}{taX labhate};\NC\NR
  \NC\NC instead of \tr{तं वहति}{taṁ vahati} we may write \tr{तव्वँहति}{taX vahai}.\NC\NR
\stoptabulate

\noindentation Or in composition,

% TODO: not correct sanskrit transcription
\startnarrower
  \tr{संयानं}{saṁyānaṁ} or \tr{सयाँनं}{saXyānaṁ};

  \tr{संलब्धं}{saṁlabdhaṁ} or \tr{सलँब्धं}{saXlabdhaṁ};

  \tr{संवहति}{saṁvahati} or \tr{सवँहति}{saXvahati}.
\stopnarrower

% second ed
\noindentation But never if the \tr{म्‌}{m} stands in the body of a word, such as
\tr{काम्यः}{kāmyaḥ}; nor if the semivowel represents an original vowel, e.g.\
Rigveda 10.\ 132, 3. \tr{सम्‌ उ आरन्‌}{sam u āran}, changed to \tr{सम्वारन्‌}{samvāran}.
%

\s The only consonants which have no corresponding nasals are \tr{र्}{r},
\tr{श्‌}{ś}, \tr{ष्‌}{ṣ}, \tr{स्‌}{s}, \tr{ह्‌}{h}. A final \tr{म्‌}{m}, therefore, before
any of these letters at the beginning of words can only be represented by the
neutral or unmodified nasal, the anusvāra.

\startnarrower
\starttabulate[|lrl|]
  \NC\tr{तं रक्षति}{taṁ rakṣati}.   \NC Or in composition,   \NC\tr{संरक्षति}{saṁrakṣati}.\NC\NR
  \NC\tr{तं श्रणोति}{taṁ śṛṇoti}.   \NC                      \NC\tr{संश्रणोति}{saṁśṛṇoti}.\NC\NR
  \NC\tr{तं षकारं}{taṁ ṣakāraṁ}.   \NC                      \NC\tr{संष्ठीवति}{saṁṣṭhīvati}.\NC\NR
  \NC\tr{तं सरति}{taṁ sarati}.    \NC                      \NC\tr{संसरति}{saṁsarati}.\NC\NR
  \NC\tr{तं हरति}{taṁ harati}.    \NC                      \NC\tr{संहरति}{saṁharati}.\NC\NR
\stoptabulate
\stopnarrower

\s In the body of a word the only letters which can be preceded by anusvāra are
\tr{श्}{ś}, \tr{ष्}{ṣ}, \tr{स्}{s}, \tr{ह्}{h}. Thus \tr{अंशः}{aṁśaḥ},
\tr{धनूंषि}{dhanūṁṣi}, \tr{यशांसि}{yaśāṁsi}, \tr{सिंहः}{siṁhaḥ}. Before the
semivowels \tr{य्}{y}, \tr{र्}{r}, \tr{ल्}{l}, \tr{व्}{v}, the \tr{म्}{m} in the
body of a word is never changed into anusvāra. Thus \tr{गम्यते}{gamyate},
\tr{नम्रः}{namraḥ}, \tr{अम्लः}{amlaḥ}.
% second ed
% As to \tr{म्}{m} before semivowels in the middle of compounds, see §9.
In \tr{शंयोः}{śaṁyoḥ} (Rv.\ 1.\ 43, 4, \&c.) the \tr{ं}{ṁ} stands
\quote{padānte,} but not in \tr{शाम्यति}{śāmyati}. (See §9.)
%

% TODO: jihvamuliya and upadhmaniya signs
\s With the exception of \tl{jihvāmūlīya} X χ (\eng{tongue-root letter}),
\tl{upadhmānīya} X φ (\eng{to be breathed on}), anusvāra \tr{ं}{ṁ}
(\eng{after-sound}), visarga \tr{ः}{ḥ} (\eng{emission}, see Taitt.-Brāhm.\ iii.\
p.\ 23 a), and \tl{repha r} (\eng{burring}), all letters are named in Sanskrit
by adding \tl{kāra} (\eng{making}) to their sounds. Thus \tr{अ}{a} is called
\tr{अकारः}{akāraḥ}; \tr{क}{ka}, \tr{ककारः}{kakāraḥ}.

\s The vowels, if initial, are written,

\starttabulate[|*{14}{p(2ex)l}|]
  \NC{\ss अ},\NC{\ss आ},\NC{\ss इ},\NC{\ss ई},\NC{\ss ऋ},\NC{\ss ॠ},\NC{\ss ऌ},\NC{\ss ॡ},\NC{\ss उ},\NC{\ss ऊ},\NC{\ss ए},\NC{\ss ऐ},\NC{\ss ओ},\NC{\ss औ};\NC\NR
  \NC\tl{a},\NC\tl{ā},\NC\tl{i},\NC\tl{ī},\NC\tl{ṛ},\NC\tl{ṝ},\NC\tl{ḷ},\NC\tl{ḹ},\NC\tl{u},\NC\tl{ū},\NC\tl{e},\NC\tl{ai},\NC\tl{o},\NC\tl{au};\NC\NR
\stoptabulate

\noindentation if they follow a consonant, they are written with the
following signs---

\starttabulate[|*{14}{p(2ex)l}|]
  \NC{\ss -},\NC{\ss ा},\NC{\ss ि},\NC{\ss ी},\NC{\ss ृ},\NC{\ss ॄ},\NC{\ss ॢ},\NC{\ss ॣ},\NC{\ss ु},\NC{\ss ू},\NC{\ss े},\NC{\ss ै},\NC{\ss ो},\NC{\ss ौ}.\NC\NR
  \NC\tl{a},\NC\tl{ā},\NC\tl{i},\NC\tl{ī},\NC\tl{ṛ},\NC\tl{ṝ},\NC\tl{ḷ},\NC\tl{ḹ},\NC\tl{u},\NC\tl{ū},\NC\tl{e},\NC\tl{ai},\NC\tl{o},\NC\tl{au}.\NC\NR
\stoptabulate

\noindentation There is one exception. If the vowel \tr{ऋ}{ṛ} follows
the consonant \tr{र्}{r}, it retains its initial form, and the \tl{r} is
written over it. Ex.\ \tr{निरृतिः}{nirṛtiḥ}.

In certain words which tolerate an hiatus in the body of a word, the second
vowel is written in its initial form. Ex.\ \tr{गोअग्र}{goagra}, adj.\
\eng{preceded by cows}, instead of \tr{गोऽग्र}{go'gra} or \tr{गवाग्र}{gavāgra};
\tr{गोअश्वं}{goaśvaṃ}, \eng{cows and horses}; \tr{प्रउग}{praüga}, \eng{yoke};
\tr{तितउ}{titaü}, \eng{sieve}.

\s Every consonant, if written by itself, is supposed to be followed by a short
\tl{a}. Thus {\ss क} is not pronounced \tl{k}, but \tl{ka}; {\ss य} not \tl{y},
but \tl{ya}. But \tr{क}{k} or any other consonant, if followed by any vowel
except \tl{a}, is pronounced without the inherent \tl{a}. Thus

\startnarrower
  \tr{का}{kā}, \tr{कि}{ki}, \tr{की}{kī}, \tr{कृ}{kṛ}, \tr{कॄ}{kṝ}, \tr{कॢ}{kḷ},
  (\tr{कॣ}{kḹ}), \tr{कु}{ku}, \tr{कू}{kū}, \tr{के}{ke}, \tr{कै}{kai}, \tr{को}{ko},
  \tr{कौ}{kau}.
\stopnarrower

% TODO: old i signs
\noindentation The only peculiarity is that short \tr{ि}{i} is apparently written before the
consonant after which it is sounded. This arose from the fact that in the
earliest forms of the Indian alphabet the long and short \tl{i}'s were both
written over the consonant, the short \tl{i} inclining to the left, the long
\tl{i} inclining to the right. Afterwards these top-marks were, for the sake of
distinctness, drawn across the top-line, so as to become {\ss कि} and {\ss की},
instead of X and X. (See Prinsep's {\em Indian Antiquities}, ed.\ Thomas,
vol.\ ii.\ p.\ 40.)

\s If a consonant is to be pronounced without any vowel after it, the consonant
is said to be followed by \tl{virāma}, i.e.\ \eng{stoppage}, which is marked by
{\ss ्}. Thus \tl{ak} must be written {\ss अक्}; \tl{kar}, {\ss कर्}; \tl{ik},
{\ss इक्}.

\s If a consonant is followed immediately by another consonant, the two or three
or four or five or more consonants are written in one group (\tl{saṁyoga}). Thus
\tl{akta} is written {\ss अत्क}; \tl{alpa} is written {\ss अल्प}; \tl{kārtsnya}
is written {\ss कार्त्स्न्य}. These groups or compound consonants must be learnt
by practice. It is easy, however, to discover some general laws in their
formation. Thus the perpendicular and horizontal lines are generally dropt in
one of the letters: {\ss क्} + {\ss क} = \tr{क्क}{kka}; {\ss न्} + {\ss द} =
\tr{न्द}{nda}; {\ss त्} + {\ss व} = \tr{त्व}{tva}; {\ss स्} + {\ss ख} =
\tr{स्ख}{skha}; {\ss च्} + {\ss य} = \tr{च्य}{cya}; {\ss प्} + {\ss त} =
\tr{प्त}{pta}; {\ss क्} + {\ss त} = \tr{क्त}{kta}; {\ss क्} + {\ss त्} + {\ss व} =
\tr{क्त्व}{ktva}; {\ss क्} + {\ss त्} + {\ss य} = \tr{क्त्य}{ktya}.

% TODO: lone r symbol
\s The \tr{र्}{r} preceding a consonant is written by XXX placed at the top of
the consonant before which it is to be sounded. Thus {\ss अर्} + {\ss क} =
\tr{अर्क}{arka}; {\ss वर्} + {\ss ष्} + {\ss म} = \tr{वर्ष्म}{varṣma}. This sign for
\tr{र}{r} is placed to the right of any other marks at the top of the same
letter. Ex.\ \tr{अर्कं}{arkaṁ}; \tr{अर्केण}{arkeṇa}; \tr{अर्केन्दू}{arkendū}.

\startnarrower
  % TODO: alternative .s sign
  \tr{क्}{k} followed by \tr{ष्}{ṣ} is written {\ss क्ष} or XXX \tl{kṣa}.

  \tr{ज्}{j} followed by \tr{ञ}{ñ} is written \tr{ज्ञ}{jña}.

  % TODO: alternative jh sign
  \tr{झ}{jh} is sometimes written XXX.

  \tr{र्}{r} followed by \tr{उ}{u} and \tr{ऊ}{ū} is written \tr{रु}{ru}, \tr{रू}{rū}.

  \tr{द्}{d} followed by \tr{उ}{u} and \tr{ऊ}{ū} is written \tr{दु}{du}, \tr{दू}{dū}.

  % TODO: half ś sign
  \tr{श}{ś}, particularly in combination with other letters, is frequently
  written XXX. Ex.\ \tr{शु}{śu}; \tr{शू}{śū}; \tr{श्र}{śra}.
\stopnarrower

\s The sign of virāma {\ss ्} (\eng{stoppage}), which if placed at the foot of a
consonant, shows that its inherent short \tl{a} is stopped, is sometimes, when
it is difficult to write (or to print) two or three consonants in one group,
placed after one of the consonants: thus {\ss युङ्क्ते} instead of
\tr{युङ्क्ते}{yuṅkte}.

\s The proper use of the virāma, however, is at the end of a sentence, or
portion of a sentence, the last word of which ends in a consonant.

At the end of a sentence, or of a half-verse, the sign {\ss ।} is used; at the
end of a verse, or of a longer sentence, the sign {\ss ।।}.

\s The sign {\ss ऽ} (\tl{avagraha} or \tl{arddhākāra}) is used in most editions
to mark the elision of an initial \tr{अ}{a}, after a final \tr{ओ}{o} or
\tr{ए}{e}. Ex.\ \tr{सोऽपि}{so'pi} for \tr{सो अपि}{so api}, i.e.\ \tr{सस्
अपि}{sas api}; \tr{तेऽपि}{te'pi} for \tr{ते अपि}{te api}.

\subject[conjuncts]{List of Compound Consonants}

% TODO: some of these conjuncts do not appear
% $kka$~\tl{k-ka}, $kkha$~\tl{k-kha}, $kca$~\tl{k-ca}, $kta$~\tl{k-ta},
% $ktya$~\tl{k-t-ya}, $ktra$~\tl{k-t-ra}, $ktrya$~\tl{k-t-r-ya},
% $ktva$~\tl{k-t-va}, $kna$~\tl{k-na}, $knya$~\tl{k-n-ya},
% $kma$~\tl{k-ma}, $kya$~\tl{k-ya}, %keep space
% % TODO: alternative k-ra
% $kra$ or XXX~\tl{k-ra}, %keep space
% % TODO: alternative k-r-ya
% $krya$~\tl{k-r-ya}, %keep space
% $kla$~\tl{k-la}, $kva$~\tl{k-va}, $kvya$~\tl{k-v-ya}, $k.sa$~\tl{k-ṣa},
% $k.sma$~\tl{k-ṣ-ma}, $k.sya$~\tl{k-ṣ-ya}, $k.sva$~\tl{k-ṣ-va};—%
% %
% $khya$~\tl{kh-ya}, $khra$~\tl{kh-ra};—%
% %
% $gya$~\tl{g-ya}, $gra$~\tl{g-ra}, $grya$~\tl{g-r-ya};—%
% %
% $ghna$~\tl{gh-na}; $ghnya$~\tl{gh-n-ya}, $ghma$~\tl{gh-ma},
% $ghya$~\tl{gh-ya}, $ghra$~\tl{gh-ra};—%
% %
% $"nka$~\tl{ṅ-ka}, $"nkta$~\tl{ṅ-k-ta}, $"nktya$~\tl{ṅ-k-t-ya},
% $"nkya$~\tl{ṅ-k-ya}, $"nk.sa$~\tl{ṅ-k-ṣa}, $"nk.sva$~\tl{ṅ-k-ṣ-va},
% $"nkha$~\tl{ṅ-kha}, $"nkhya$~\tl{ṅ-kh-ya}, $"nga$~\tl{ṅ-ga},
% $"ngya$~\tl{ṅ-g-ya}, $"ngha$~\tl{ṅ-gha}, $"nghya$~\tl{ṅ-gh-ya},
% $"nghra$~\tl{ṅ-gh-ra}, $"n"na$~\tl{ṅ-ṅa}, $"nma$~\tl{ṅ-ma},
% $"nya$~\tl{ṅ-ya}.

% TODO: finish the above

\subject[numbers]{Numerical Figures}

\s The numerical figures in Sanskrit are

\startalignment[middle]
  \dontleavehmode
  \vbox{%
  \starttabulate[|k1ck1ck1ck1ck1ck1ck1ck1ck1ck1c|]
    \NC{\ss १} \NC{\ss २} \NC{\ss ३} \NC{\ss ४} \NC{\ss ५} \NC{\ss ६} \NC{\ss ७} \NC{\ss ८} \NC{\ss ९} \NC{\ss ०} \NC\NR
    \NC 1 \NC 2 \NC 3 \NC 4 \NC 5 \NC 6 \NC 7 \NC 8 \NC 9 \NC 0 \NC\NR
  \stoptabulate
  }
\stopalignment

\startnotes
\noindentation These figures were originally abbreviations of the initial letters of the
Sanskrit numerals. The Arabs, who adopted them from the Hindus, called them
Indian figures; in Europe, where they were introduced by the Arabs, they were
called Arabic figures.

\startnarrower
\starttabulate
  \NC Thus \NC {\ss १} stands for \tr{ए}{e} of \tr{एकः}{ekaḥ}, \eng{one}.\NC\NR
  \NC      \NC {\ss २} stands for \tr{द्व}{dv} of \tr{द्वौ}{dvau}, \eng{two}.\NC\NR
  \NC      \NC {\ss ३} stands for \tr{त्र}{tr} of \tr{त्रयः}{trayaḥ}, \eng{three}.\NC\NR
  \NC      \NC {\ss ४} stands for \tr{च}{c} of \tr{चत्वारः}{catvāraḥ}, \eng{four}.\NC\NR
  \NC      \NC {\ss ५} stands for \tr{प}{p} of \tr{पञ्च}{pañca}, \eng{five}.\NC\NR
\stoptabulate
\stopnarrower

\noindentation The similarity becomes more evident by comparing the letters and numerals as
used in ancient inscriptions. See Woepcke, {\em Mémoire sur la Propagation des
Chiffres Indiens}, in {\em Journal Asiatique}, {\sc vi} série, tome {\sc i};
Prinsep's {\em Indian Antiquities by Thomas}, vol.\ {\sc ii}.\ p.\ 70; {\em
Chips from a German Workshop}, vol.\ {\sc ii}.\ p.\ 289.
\stopnotes

\subject[pronunciation]{Pronunciation.}

\s The Sanskrit letters should be pronounced in accordance with the
transcription given page~4. The following rules, however, are to be observed:

\startitemize[n]
  \item The vowels should be pronounced like the vowels in Italian. The short
  \tr{अ}{a}, however, has rather the sound of the English {\em a} in
  \quote{America.}

  \item The aspiration of the consonants should be heard distinctly. Thus
  \tr{ख}{kh} is said, by English scholars who have learnt Sanskrit in India, to
  sound almost like {\em kh} in \quote{inkhorn;} \tr{थ}{th} like {\em th} in
  \quote{pothouse;} \tr{फ}{ph} like {\em ph} in \quote{topheavy;} \tr{घ}{gh}
  like {\em gh} in \quote{loghouse;} \tr{ध}{dh} like {\em dh} in
  \quote{madhouse;} \tr{भ}{bh} like {\em bh} in \quote{Hobhouse.} This, no
  doubt, is a somewhat exaggerated description, but it is well in learning
  Sanskrit to distinguish from the first the aspirated from the unaspirated
  letters by pronouncing the former with an unmistakable emphasis.

  \item The guttural \tr{ङ}{ṅ} has the sound of {\em ng} in \quote{king.}

  \item The palatal letters \tr{च}{c} and \tr{ज}{j} have the sound of {\em ch}
  in \quote{church} and of {\em j} in \quote{join.}

  \item The lingual letters are said to be pronounced by bringing the lower
  surface of the tongue against the roof of the palate. As a matter of fact the
  ordinary pronunciation of {\em t}, {\em d}, {\em n} in English is what Hindus
  would call lingual, and it is essential to distinguish the Sanskrit dentals by
  bringing the tip of the tongue against the very edge of the upper front-teeth.
  In transcribing English words the natives naturally represent the English
  dentals by their linguals, not by their own dentals; e.g.\
  \tr{डिरेक्टर्}{Ḍirekṭar}, \tr{गवर्ण्मेणट्}{Gavarṇmeṇṭ}, \&c.\footnote{Bühler, {\em
  Madras Literary Journal}, February, 1864. Rajendralal Mitra, {\em On the
  Origin of the Hindvī Language}, {\em Journal of the Asiatic Society}, Bengal,
  1864, p.\ 509.}

  \item The visarga, jihvāmūlīya and upadhmānīya are not now articulated
  audibly.

  \item The dental \tr{स}{s} sounds like {\em s} in \quote{sin,} the lingual
  \tr{ष}{ṣ} like {\em sh} in \quote{shun,} the palatal \tr{श}{ś} like {\em ss}
  in \quote{session.}
\stopitemize

The real anusvāra is sounded as a very slight nasal, like {\em n} in French
\quote{bon.} If the dot is used as a graphic sign in place of the other five
nasals it must, of course, be pronounced like the nasal which it
represents.\footnote{According to Sanskrit grammarians the real anusvāra is
pronounced in the nose only, the five nasals by their respective organs and the
nose. Siddh.-Kaum.\ to Pāṇini 1.1.9.
{\ss ञमङणनानां नासिका च (चकारेण खखवर्गोच्चारानुकूलं ताल्वादि समुच्चीयते)।।} The real anusvāra is
therefore \tl{nāsikya}, \eng{nasal}; the five nasals are \tl{anunāsika},
\eng{nasalized}, i.e.\ pronounced by their own organ of speech, and uttered
through the nose.}
\stopcomponent
